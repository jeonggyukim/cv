%% start of file `template.tex'.
%% Copyright 2006-2013 Xavier Danaux (xdanaux@gmail.com).
%
% This work may be distributed and/or modified under the
% conditions of the LaTeX Project Public License version 1.3c,
% available at http://www.latex-project.org/lppl/.

\documentclass[11pt,letterpaper,roman]{moderncv}        % possible options include font size ('10pt', '11pt' and '12pt'), paper size ('a4paper', 'letterpaper', 'a5paper', 'legalpaper', 'executivepaper' and 'landscape') and font family ('sans' and 'roman')

% moderncv themes
%\moderncvstyle{banking}                 % style options are 'casual' (default), 'classic', 'oldstyle' and 'banking'
\moderncvstyle{banking}                 % style options are 'casual' (default), 'classic', 'oldstyle' and 'banking'
\moderncvcolor{blue}                    % color options 'blue' (default), 'orange', 'green', 'red', 'purple', 'grey' and 'black'
%\renewcommand{\familydefault}{\sfdefault}         % to set the default font; use '\sfdefault' for the default sans serif font, '\rmdefault' for the default roman one, or any tex font name
\nopagenumbers{}                      % uncomment to suppress automatic page numbering for CVs longer than one page

\usepackage{xcolor}

\usepackage{enumitem}
\setlist{leftmargin=7.5mm}
%\setlist[enumerate]{itemsep=1mm}

\definecolor{babyblue}{rgb}{0.54, 0.81, 0.94}
\definecolor{dimgray}{rgb}{0.41, 0.41, 0.41}
\definecolor{dodgerblue}{rgb}{0.12, 0.56, 1.0}

\newenvironment{benumerate}[1]{
  \let\oldItem\item
  \def\item{\addtocounter{enumi}{-2}\oldItem}
  \begin{enumerate}[itemsep=0.0mm]
    \setcounter{enumi}{#1}
    \addtocounter{enumi}{1}
  }{
  \end{enumerate}
}


% character encoding
\usepackage[utf8]{inputenc}            % if you are not using xelatex ou lualatex, replace by the encoding you are using
%\usepackage{CJKutf8}                  % if you need to use CJK to typeset your resume in Chinese, Japanese or Korean

% adjust the page margins
\usepackage[scale=0.75]{geometry}
%\setlength{\hintscolumnwidth}{3cm}                % if you want to change the width of the column with the dates
%\setlength{\makecvtitlenamewidth}{10cm}           % for the 'classic' style, if you want to force the width allocated to your name and avoid line breaks. be careful though, the length is normally calculated to avoid any overlap with your personal info; use this at your own typographical risks...

% personal data
\name{Jeong-Gyu}{Kim}
\title{Curriculum Vitae}                            % optional, remove / comment the line if not wanted
\address{Division of Science}{2-21-1, Osawa, Mitaka, Tokyo 181-0015 Japan}
% \address{Theoretical Astronomy Department, 308 Lee Wonchul Hall}{776, Daedeokdae-ro, Yuseong-gu, Daejeon 34055, Republic of Korea}
%\address{Department of Astrophysical Sciences, 4 Ivy Lane}{Princeton University,
%  Princeton}{NJ 08544, USA}
% \address{Department of Physics and Astronomy, 1 Gwanak-ro, Gwanak-gu}{Seoul, 08826}{Republic of Korea} % optional, remove / comment the line if not wanted; the "postcode city" and and "country" arguments can be omitted or provided empty
% \phone[mobile]{+1~(609)~933~8470}                   % optional, remove / comment the line if not wanted
% \phone[mobile]{+82~10~8638~0722}                   % optional, remove / comment the line if not wanted
%\phone[fixed]{+2~(345)~678~901}                    % optional, remove / comment the line if not wanted
%\phone[fax]{+3~(456)~789~012}                      % optional, remove / comment the line if not wanted
\email{jeonggyu.astro@gmail.com}                    % optional, remove / comment the line if not wanted
% \email{jgkim@kasi.re.kr}                    % optional, remove / comment the line if not wanted
%\email{kimjg@astro.princeton.edu}                  % optional, remove / comment the line if not wanted
\homepage{jeonggyukim.github.io}                   % optional, remove / comment the line if not wanted
%\extrainfo{additional information}                 % optional, remove / comment the line if not wanted
%\photo[64pt][0.4pt]{picture}                       % optional, remove / comment the line if not wanted; '64pt' is the height the picture must be resized to, 0.4pt is the thickness of the frame around it (put it to 0pt for no frame) and 'picture' is the name of the picture file
%\quote{}                                 % optional, remove / comment the line if not wanted

% to show numerical labels in the bibliography (default is to show no labels); only useful if you make citations in your resume
%\makeatletter
%\renewcommand*{\bibliographyitemlabel}{\@biblabel{\arabic{enumiv}}}
%\makeatother
%\renewcommand*{\bibliographyitemlabel}{[\arabic{enumiv}]}% CONSIDER REPLACING THE ABOVE BY THIS

% bibliography with mutiple entries
%\usepackage{multibib}
%\newcites{book,misc}{{Books},{Others}}
%----------------------------------------------------------------------------------
%            content
%----------------------------------------------------------------------------------
\begin{document}
%\begin{CJK*}{UTF8}{gbsn}                          % to typeset your resume in Chinese using CJK
%-----       resume       ---------------------------------------------------------
\makecvtitle

\section{Employment}
% \cventry{10/2022--present}{National Astronomical Observatory of Japan}{EACOA Fellow}{Mitaka, Tokyo, Japan}{}{
\cventry{10/2022--}{National Astronomical Observatory of Japan}{EACOA Fellow}{Mitaka, Japan}{}{
%\cventry{10/2022--present}{National Astronomical Observatory of Japan}{EACOA Fellow}{Mitaka, Tokyo, Japan}{}{
%   % \textit{Mentor}: Prof. Fumitaka Nakamura \& Prof. Thiem Hoang
}

%\cventry{10/2021--09/2022}{Korea Astronomy and Space Science Institute}{EACOA Fellow}{Daejeon, Korea}{}{
\cventry{10/2021--09/2022}{Korea Astronomy and Space Science Institute}{EACOA Fellow}{Daejeon, Korea}{}{
 % \textit{Mentor}: Prof. Thiem Hoang
}

% \cventry{09/2018--08/2021}{Department of Astrophysical Sciences, Princeton
\cventry{09/2018--08/2021}{Department of Astrophysical Sciences, Princeton
  University}{Lyman Spitzer, Jr. Postdoctoral Fellow}{Princeton, NJ, USA}{}{
  % \textit{Mentor}: Prof. Eve Ostriker
}

% \cventry{02/2012}{M.S. in Astronomy}{Seoul National University}{Seoul,
%   Korea}{}{\textit{Advisor}: Prof. Woong-Tae Kim
%   \\
%   \textit{Thesis:} Gravitational Instability of Vertically-stratified,
%   Pressure-confined, Rotating, Polytropic
%   Disks} % arguments 3 to 6 can be left empty

% \cventry{02/2010}{B.S. in Astronomy, \textit{cum laude}}{Seoul National
%   University}{Seoul, Korea}{}{}{}

% \section{Master thesis}
% \cvitem{Title}{\emph{Title}}
% \cvitem{supervisors}{Supervisors}
% \cvitem{description}{Short thesis abstract}

\section{Education and Research Experience}
% \mycventry{Seoul National University}{Seoul, Korea}{Ph.D. in Astronomy}{}{2018}{aa}{bb}{cc}

\begin{tabular*}{\textwidth}{l@{\extracolsep{\fill}}r}
  {\bfseries Seoul National University} & {\bfseries Seoul, Korea} \\
  %{Ph.D. in Astronomy (Advisor: Prof. Woong-Tae Kim)} & {\itshape 08/2018}\\%
  {Ph.D. in Astronomy (Advisor: Prof. Woong-Tae Kim)} & {\itshape 08/2018}\\%
  % {\small\textit{Advisor}: Prof. Woong-Tae Kim} \\
  % {\small\textit{Thesis Title:} Dynamical Evolution of Giant Molecular Clouds Driven by}\\
  % \hspace{0.72in}{\small UV Radiation Feedback from Massive Stars} \\
  \rule[-1.2ex]{-2.5pt}{4ex}

%  {M.S. in Astronomy} & {\itshape 02/2012}\\%
  {M.S. in Astronomy} & {\itshape 02/2012}\\%
  % {\small\textit{Thesis:} Gravitational Instability of
  % Vertically-stratified, Pressure-confined, Rotating, Polytropic
  % Disks}\\
  \rule[-1.2ex]{-2.5pt}{4ex}

  % {\itshape B.S. in Astronomy {\small\textit{(cumulative GPA: 3.65/4.3, major GPA: 3.97/4.3; cum laude)}}} & {\itshape 02/2010}\\%
%  {B.S. in Astronomy} & {\itshape 02/2010}\\%
  {B.S. in Astronomy} & {\itshape 02/2010}\\%
\end{tabular*}%

\cventry{2014--2016 (2mo/yr)}{Visiting Student Research Collaborator (\textit{Mentor}: Prof. Eve Ostriker)}{Princeton University}{Princeton, NJ, USA}{}{}

% \cventry{08/2018, expected}{Ph.D. in Astronomy}{Seoul National
%   University}{Seoul, Korea}{}{\textit{Advisor}: Prof. Woong-Tae Kim
%   \\
%   \textit{Thesis:} Disruption of Molecular Clouds by Radiative
%   Feedback from Massive Stars} % arguments 3 to 6 can be left empty
%  \rule[-0.9ex]{-2.5pt}{2ex}

\section{Honors and Fellowships}
\begin{tabular*}{\textwidth}{l@{\extracolsep{\fill}}r}
  {{\bfseries EACOA Fellowship}, East Asian Core Observatory Association} & {2021--2024} \\
  \rule[-1.2ex]{-2.5pt}{4ex}
  {{\bfseries Lyman Spitzer, Jr. Fellowship}, Princeton University} & {2018--2021} \\
  \rule[-1.2ex]{-2.5pt}{4ex}
  {{\bfseries Outstanding Thesis Award}, Seoul National University} & {2018} \\
  \rule[-1.2ex]{-2.5pt}{4ex}
  {{\bfseries National Junior Research Fellowship}, NRF} & {2014--2018} \\
  \rule[-1.2ex]{-2.5pt}{4ex}
  {{\bfseries Fellowship for the Next Generation of Basic Research}, SNU} & {2013} \\
  \rule[-1.2ex]{-2.5pt}{4ex}
  {{\bfseries Scholarship for Superior Academic Performance}, Brain Korea 21} & {2012} \\
  \rule[-1.2ex]{-2.5pt}{4ex}
  {{\bfseries Lotte Scholarship}, Lotte Foundation} & {2010--2011} \\
  \rule[-1.2ex]{-2.5pt}{4ex}
  {{\bfseries National Scholarship For Science and Engineering}, NRF} & {2003--2004, 2009--2010}
\end{tabular*}%

% \cventry{2021--2024}{East Asian Core Observatory Association}{EACOA Fellowship}{}{}{}

% \cventry{2018--2021}{Princeton University}{Lyman Spitzer, Jr. Fellowship}{}{}{}

% \cventry{2018}{College of Natural Sciences, Seoul National University}{Outstanding Thesis Award}{}{}{}

% \cventry{2014--2018}{National Research Foundation of Korea}{National
%   Junior Research Fellowship}{}{}{}

% % \cventry{2014--2018}{National Research Foundation of Korea}{National
% %   Junior Research Fellowship (\normalfont{Grants obtained as PI: \$41,300/\emph{yr}})}{}{}{Project name: \textit{``Expansion of
% %     Dusty Magnetized H II Regions and Their Dynamic Impact on the
% %     Interstellar Medium''}}

% \cventry{2013}{Seoul National University}{SNU Fellowship for the
%   Next Generation of Basic Research}{}{}{}

% % \cventry{2013}{Department of Physics and Astronomy, Seoul National
% %   University}{SNU Development Fund (Inha Kim) Scholarship}{}{}{}

% \cventry{2012}{Brain Korea 21}{Scholarship for Superior Academic
%   Performance}{}{}{}

% \cventry{2010--2011}{Full tuition awarded by Lotte Foundation}{Lotte
%   Scholarship}{}{}{}

% \cventry{2003--2004, 2008--2009}{Full tuition awarded by National
%   Research Foundation of Korea}{National Scholarship For Science and
%   Engineering}{}{}{}

% Publications from a BibTeX file without multibib
%  for numerical labels:
%  \renewcommand{\bibliographyitemlabel}{\@biblabel{\arabic{enumiv}}}%
%  CONSIDER MERGING WITH PREAMBLE PART to redefine the heading string ("Publications"): \renewcommand{\refname}{Articles}
% \nocite{*}
% \bibliographystyle{myplain}
% \bibliography{publications}                        % 'publications' is the name of a BibTeX file

\section{Research Interests}

Stellar feedback, star formation, HII regions, lifecycle of molecular clouds, dynamics and thermodynamics of the interstellar medium, physics of grain alignment

\section{Advising Experience}
\begin{itemize}
\setlength\itemsep{0.0em}
\item \textbf{Nora Linzer}, Princeton PhD student, \textit{Interstellar UV
    radiation field in TIGRESS simulations} (semester project), co-advised with
  Prof. Eve Ostriker and Dr. Chang-Goo Kim, 2021--
\item \textbf{Nguyen Chau Giang}, PhD student in University of Science \&
  Technology, \textit{Modeling Polarized Thermal Emission from Dust Grains in
    Protostellar Cores} (PhD Project), co-advised with Prof. Thiem Hoang,
  2021--2022
\item \textbf{Lachlan Lancaster}, Princeton PhD student, \textit{Stellar Wind
    Bubble Expansion in the Turbulent ISM} (thesis project), co-advised with
  Prof. Eve Ostriker and Dr. Chang-Goo Kim, 2019--2022
\item \textbf{Erin Kado-Fong}, Princeton PhD student, \textit{Diffuse Ionized Gas in
    Simulations of Multiphase, Star-forming Galactic Disks} (semester project),
  co-advised with Prof. Eve Ostriker and Dr. Chang-Goo Kim, 2018--2020
\item \textbf{Nina Filippova}, Princeton undergraduate, \textit{Numerical
    Magnetohydrodynamics Simulations of Star Formation and Giant Molecular Cloud
    Destruction} (senior thesis), co-advised with Prof. Eve Ostriker, 2019--2020
\end{itemize}


% Comment out to omit
\section{Publications}

\subsection{Journal Publications -- \href{https://ui.adsabs.harvard.edu/search/filter_database_fq_database=OR&filter_database_fq_database=database\%3A\%22astronomy\%22&fq=\%7B!type\%3Daqp\%20v\%3D\%24fq_database\%7D&fq_database=(database\%3A\%22astronomy\%22)&q=\%3Dauthor\%3A(\%22Kim\%2C\%20Jeong-Gyu\%22)&sort=date\%20desc\%2C\%20bibcode\%20desc}{\color{dodgerblue}{ADS
Search}}}

% ** Important publications ;

\# Led by a student under direct supervision

\begin{benumerate}{18}
\item \textit{Physical Modeling of Dust Polarization from Magnetically
Enhanced Radiative Torque (MRAT) Alignment in Protostellar Cores with POLARIS}
\\ Giang, N. C., Hoang, T., \textbf{Kim, J.-G.}, \& Tram, L. N. \textit{2022,
MNRAS submitted}
\item \textit{Photochemistry and Heating/Cooling of the Multiphase Interstellar
Medium with UV Radiative Transfer in Magnetohydrodynamic Simulations} \\
\textbf{Kim, J.-G.}, Gong, M., Kim, C.-G., \& Ostriker, E. C., \textit{ApJS
submitted}
\item \textit{Slow Star Formation in the Milky Way: Theory Meets Observations}\\
  Evans, Neal J., II, \textbf{Kim, J.-G.}, \& Ostriker, E. C. 2022, ApJL, 929L,
  18E
\item \textit{Star Formation Regulation and Self-Pollution by Stellar Wind
    Feedback}\\
  Lancaster, L., Ostriker, E. C., \textbf{Kim, J.-G.}, \& Kim, C.-G. 2021, ApJL,
  922L, 3L
\item \textit{Efficiently Cooled Stellar Wind Bubbles in Turbulent Clouds:
    II. Validation of Theory with Hydrodynamic Simulations}\\
  Lancaster, L., Ostriker, E. C., \textbf{Kim, J.-G.}, \& Kim, C.-G. 2021, ApJ,
  914, 90L
\item \textit{Efficiently Cooled Stellar Wind Bubbles in Turbulent Clouds:
I. Fractal Theory and Application to Star-Forming Clouds}\\
  Lancaster, L., Ostriker, E. C., \textbf{Kim, J.-G.}, \& Kim, C.-G., 2021, ApJ,
914, 89L
\item \textit{Star Formation Efficiency and Dispersal of Giant Molecular
    Clouds with UV Radiation Feedback: Dependence on Gravitational Boundedness
    and Magnetic Fields} \\ \textbf{Kim, J.-G.}, Ostriker, E. C., \& Filippova,
  N. 2021, ApJ, 911, 128K
\item \textit{The environmental dependence of the $X_{\mathrm CO}$ conversion
    factor}\\ Gong, M., Ostriker, E. C., Kim, C.-G., \& \textbf{Kim,
    J.-G.} 2020, ApJ, 903, 142 
\item \textit{Factories of CO-dark gas: molecular clouds with limited star
formation efficiencies by far-ultraviolet feedback} \\
  Inoguchi, M., Hosokawa, T., Mineshige, S., \& \textbf{Kim, J.-G.} 2020, MNRAS,
  497, 5061I
\item \textit{\# Diffuse Ionized Gas in Simulations of Multiphase, Star-forming
    Galactic Disks}\\
  Kado-Fong, E., \textbf{Kim, J.-G.}, Ostriker, E. C., \& Kim, C.-G. 2020, ApJ,
  897, 143
\item \textit{Modeling UV Radiative Feedback from Massive Stars: III. Escape
    of Radiation from Star-Forming Giant Molecular Clouds} \\ \textbf{Kim,
    J.-G.}, Kim, W.-T., \& Ostriker, E. C. 2019, ApJ, 883, 102
\item \textit{Modeling UV Radiative Feedback from Massive Stars: II. Dispersal
of Star-Forming Giant Molecular Clouds by Photoionization and Radiation
Pressure} \\ \textbf{Kim, J.-G.}, Kim, W.-T., \& Ostriker, E. C. 2018, ApJ, 859,
68
\item \textit{Modeling UV Radiative Feedback from Massive Stars: I.
Implementation of Adaptive Ray Tracing Method and Tests}\\ \textbf{Kim, J.-G.},
Kim, W.-T., Ostriker, E. C.,\& Skinner A. M. 2017, ApJ, 851, 93
\item \textit{Disruption of Molecular Clouds by Expansion of Dusty H II
Regions}\\ \textbf{Kim, J.-G.}, Kim, W.-T., \& Ostriker, E. C. 2016, ApJ, 819,
137
\item \textit{Instability of Magnetized Ionization Fronts Surrounding H II
regions}\\ \textbf{Kim, J.-G.}, Kim, W.-T. 2014, ApJ, 797, 135
\item \textit{Nature of Wiggle Instability of Galactic Spiral Shocks}\\ Kim,
W.-T., Kim, Y., \& \textbf{Kim, J.-G.} 2014, ApJ, 789, 68
\item \textit{Instability of Evaporation Fronts in the Interstellar Medium}\\
  \textbf{Kim, J.-G.}, Kim, W.-T. 2013, ApJ, 779, 48
\item \textit{Gravitational Instability of Rotating, Pressure-Confined,
    Polytropic Gas Disks with Vertical Stratification} \\ \textbf{Kim, J.-G.},
  Kim, W.-T., Seo Y. M., \& Hong, S. S. 2012, ApJ, 761, 131
\end{benumerate}

\subsection{Papers in Preparation}
\begin{enumerate}
\item[] \textit{What Regulates Galactic Star Formation Rates? A View from New
    TIGRESS Simulations} \\ Kim, C.-G., \textbf{Kim, J.-G.}, Gong, M., \& Ostriker,
  E. C. \textit{2022, to be submitted}
\item[] \textit{Destruction of Giant Molecular Clouds by Stellar Feedback: I.
    Relative Importance of Radiation, Winds, and Supernovae} \\ \textbf{Kim,
    J.-G.}, Lancaster, L., Kim, C.-G., \& Ostriker, E. C. \textit{2022, in
    prep.}
\item[] \textit{The Lyman Continuum and Far-UV Interstellar Radiation Field in
    Simulations of Multiphase, Star-forming Galactic Disks} \\ Linzer, N.,
  Ostriker, E. C., \textbf{Kim, J-.G.}, \& Kim, C.-G. \textit{2022, in prep.}
\item[] \textit{The Effects of Magnetic Fields and H II regions on the Dynamics
of Stellar Wind Bubbles} \\ Lancaster, L., Ostriker, E. C.,
  \textbf{Kim, J.-G.}, \& Kim, C.-G. \textit{2022, in prep.}
% \item[] \textit{r-process Enrichment During the Formation of a Globular Cluster: Case
%     of M15} \\
%   Hotokezaka, K., \textbf{Kim, J.-G.}, Beniamini, P., \& Cen, R. \textit{in prep.}
 % \item[] \textit{Recovery of Cloud-Scale Star Formation Rate Based on Ionizing
 %    Luminosity: Calibration for Escape of Radiation} \\ \textbf{Kim, J.-G.} \&
 %  Ostriker, E. C. \textit{in prep.}
\end{enumerate}

%%% Local Variables: %%% mode: latex %%% TeX-master: t %%% End:


%\section{Selected Conferences and Talks (2019-)}
\section{Conferences and Talks (2019-)}

%\begin{benumerate}{14}
\begin{itemize}
  \setlength\itemsep{0em}
\item \textbf{Contributed Talk},
  % \textit{Structure of Density-Bounded H II Regions}
  SAGI workshop, Quy Nhon, Vietnam, Nov, 2023 %
\item \textbf{Invited Talk},
  % \textit{Structure of Density-Bounded H II Regions}
  Star and Planet Formation Seminar, NAOJ (online), Mitaka, Japan, Oct, 2023 %
\item \textbf{Seminar},
  % \textit{A Model for Ionized Gas Outflows in HII Regions}
  ALMA-DS Seminar, NAOJ, Mitaka, Japan, Sep, 2023 % Sep 6
\item \textbf{Invited Seminar},
  % \textit{Numerical Modeling of Star-forming, Multiphase Interstellar Medium with Stellar Feedback}
  Nagoya Univ., Nagoya, Japan, Jul, 2023 % Jul 31
\item \textbf{Contributed Talk},
  % \textit{Star Formation Efficiency and Dispersal of Star-forming GMCs with UV Radiation Feedback}
  Interstellar Institute 6, Institut Pascal, Orsay, France, Jul, 2023 % Jul 3
\item \textbf{Contributed Talk},
  % \textit{A new model for ISM heating/cooling and its applications}
  Lyon, France, Jun, 2023 % Jun 26
\item \textbf{Colloquium},
  % \textit{Star Formation Efficiency and Dispersal of Giant Molecular Clouds with Stellar Feedback}
  Osaka Univ., Osaka, Japan, May, 2023 % May 24
\item \textbf{Contributed Talk},
  % \textit{Adaptive Ray Tracing}\\
  Athena++ Workshop 2023, Center for Computational Astrophysics, New York, USA, May, 2023 % May 11
\item \textbf{Contributed Talk},
  % \textit{Star Formation Efficiency and Dispersal of Molecular Clouds with UV Radiation Feedback}
  Workshop on Star Formation: From Clouds to Cores, NAOJ, Mitaka, Japan, Apr, 2023 % Apr 19
\item \textbf{Poster},
  Protostars and Planets VII, Kyoto, Japan, Apr, 2023 % Apr 10--15
\item \textbf{Colloquium},
  % \textit{Numerical Modeling of Star-forming, Multiphase Interstellar Medium with Stellar Feedback}
  Kyungpook Nat'l Univ., Daegu, Korea, Apr, 2023 % Apr 4
\item \textbf{Contributed Talk},
  % \textit{Introducing TIGRESS-NCR: Co-regulation of Multiphase ISM and Star Formation Rates}
  DARWIN+Numerical Galaxy Formation Joint Workshop, Konjiam, Korea, Jan, 2023 % Jan 11
\item \textbf{Contributed Talk},
  % \textit{Introducing TIGRESS-NCR: multiphase ISM structure, star formation rates, and more}
  Recent Advances in Galaxy Formation and Reionization 2022, Yonsei Univ., Seoul, Korea, Nov, 2022 % Nov 7
\item \textbf{Invited Colloquim},
  % \textit{Star formation efficiency and destruction of giant molecular clouds with stellar feedback}
  NAOJ, Mitaka, Japan, Oct, 2022 % Oct 26
\item \textbf{Invited Colloquim},
  % \textit{Numerical modeling of star formation and stellar feedback in giant molecular clouds}
  Seoul Nat'l Univ., Seoul, Korea, Sep, 2022 % Sep 15
\item \textbf{Contributed Talk},
  % \textit{Star Formation Efficiency and Destruction of Giant Molecular Clouds with UV Radiation Feedback}
  IAU Symposium 373: Resolving the Rise and Fall of Star Formation in Galaxies, Busan, Korea, Aug, 2022 % Aug 10
\item \textbf{Invited Colloquim},
  % \textit{Star Formation Efficiency and Dispersal of Giant Molecular Clouds with Stellar Feedback}
  Yonsei Univ. (online), Seoul, Korea, Apr, 2022 % Apr 29
\item \textbf{Invited Colloquim},
  % \textit{Star Formation Efficiency and Dispersal of Giant Molecular Clouds with Stellar Feedback}\\
  Chungnam Nat'l Univ., Daejeon, Korea, Apr, 2022 % Apr 21
\item \textbf{Contributed Talk},
  % \textit{Star Formation Efficiency and Dispersal of Giant Molecular Clouds with Stellar Feedback}
  KAS Spring Meeting, Busan, Korea, Apr, 2022 % Apr 14
\item \textbf{Invited Talk},
  % \textit{Modeling photochemistry and heating/cooling of the ISM with UV radiative transfer}\\
  Breakthroughs in Galaxy Formation, Ringberg Castle, Germany, Apr, 2022 % Apr 6
\item \textbf{Seminar},
  % \textit{Slow Star Formation in the Milky Way: Theory Meets Observations}\\
  Theoretical Astrophysics Group Seminar, KASI, Daejeon, Korea, Mar, 2022 % Mar 24
\item \textbf{Contributed Talk},
  % \textit{Star Formation Efficiency and Destruction of Giant Molecular Clouds with UV Radiation Feedback},
  The 1st VARNET Workshop on Star Formation and Stellar Feedback (online), Dec, 2021 % Dec 9
\item \textbf{Colloquium},
  % \textit{Star Formation Efficiency and Destruction of Giant Molecular Clouds with UV Radiation Feedback},
  KASI, Daejeon, Korea, Nov, 2021 % Nov 24
%\item \textbf{Invited Talk},
  % \textit{Modeling ISM Thermochemistry Coupled with UV Radiative Transfer}\\
%  Center for Computational Astrophysics, Flatiron Institute, NY, USA, Jun 21, 2021
\item \textbf{Invited Seminar},
  % \textit{Modeling Dispersal of Molecular Clouds by UV Radiation Feedback}\\
  CCAPP Seminar (online), Ohio State University, OH, USA, Nov, 2020 % Nov 24
\item \textbf{Invited Seminar},
  % \textit{Modeling Dispersal of Molecular Clouds by UV Radiation Feedback}\\
  Thunch Seminar (online), Princeton University, NJ, USA, Nov, 2020 % Nov 12
\item \textbf{Invited Seminar},
  % \textit{Modeling Dispersal of Molecular Clouds by UV Radiation Feedback}\\
  Astronomy Seminar (zoom), University of Kentucky, KY, USA, Oct, 2020 % Oct 29
%\item \textbf{Invited Talk},
  % \textit{Modeling Dispersal of Molecular Clouds by UV Radiation Feedback}\\
%  Ringberg Workshop on Computational Galaxy Formation, Tegernsee, Germany, Apr 20, 2020 (Cancelled due to COVID-19)
\item \textbf{Invited Review},
  %\textit{Numerical Modeling of Warm Ionized Medium: A Large-scale Perspective}\\
  WIM in Galaxies Workshop, Green Bank Observatory, WV, USA, Oct, 2019 % Oct 8
\item \textbf{Contributed Talk},
  % \textit{Modeling UV Radiation Feedback from Massive Stars}\\
  The Self-organized Star Formation Process, Institut Pascal, Orsay, France, Sep, 2019 % Sep 30
\item \textbf{Seminar},
  %\textit{Modeling UV Radiation Feedback from Massive Stars}\\
  KASI, Daejeon, Korea, Aug, 2019 % Aug 29
\item \textbf{Colloquium},
  % \textit{Modeling UV Radiation Feedback from Massive Stars}\\
  Max Planck Institute for Radio Astronomy, Bonn, Germany, Jul, 2019 % Jul 3
\item \textbf{Contributed Talk},
  % \textit{Diffuse Ionized Gas in TIGRESS Simulations of the ISM}\\
  European Week of Astronomy \& Space Science 2019, Lyon, France, Jun, 2019 % Jun 27
\item \textbf{Contributed Talk},
  %\textit{Dispersal of GMCs by UV Radiation Feedback from Massive Stars}\\
  Zooming in on Star Formation, Nafplio, Greece, Jun, 2019 % Jun 13
\item \textbf{Contributed Talk},
  % \textit{Adaptive Ray Tracing in {\textit Athena}}\\
  Athena++ Workshop 2019, Las Vegas, USA, Mar, 2019 % Mar 18--22
% \item \textbf{Poster},
%   % \textit{Modeling UV Radiation Feedback from Massive Stars: Dispersal of GMCs and Escape of Radiation}\\
%   van de Hulst Centennial Symposium % : The interstellar Medium of Galaxies: Status and Future Perspectives,
%   Leiden, the Netherlands, Nov, 2018 % Nov 5--9
% \item \textbf{Poster},
%   % \textit{Dispersal of Giant Molecular Clouds by UV Radiation Feedback from Massive Stars}\\
%   15th Potsdam Thinkshop
%   % : The role of feedback in galaxy formation: from small-scale winds to large-scale outflows,
%   Potsdam, Germany, Sep, 2018 % Sep 3-7
% \item \textbf{Poster},
%   % \textit{Dispersal of Giant Molecular Clouds by Photoionization and Radiation Pressure}\\
%   231st AAS Meeting, Washington D.C., USA, Jan, 2018 % Jan 11
% \item \textbf{Colloquium},
% %  \textit{Dispersal of Giant Molecular Clouds by Photoionization and Radiation Pressure}\\
%   Osaka University, Japan, Dec, 2017 % Dec 21
% \item \textbf{Contributed Talk},
%   % \textit{Dispersal of Molecular Clouds by Photoionization and Radiation Pressure}\\
%   Star Formation in Different Environments, Quy Nhon, Vietnam, Aug, 2017 % Aug 7
% \item Contributed Talk, \textit{Dispersal of Molecular Clouds by UV
%     Radiation Feedback from Massive Stars}\\ 2017 Korean Astronomical
%   Society Spring Meeting, Seoul, Korea, Apr 13, 2017
% \item Seminar Talk, \textit{Radiation Hydrodynamic Simulations of Star Cluster
%     Formation in Turbulent Molecular Clouds}\\ Korean Numerical
%   Astrophysics Group Meeting, Daejeon, Korea, Dec 16, 2016
% \item \textbf{Seminar Talk},
%   % \textit{Modeling Radiative Feedback from Massive Stars}\\
%   Star Formation/ISM Rendezvous, Princeton University, USA, Nov, 2016 % Nov 28
% \item \textbf{Poster},
%   % \textit{Disruption of Molecular Clouds by Radiative Feedback from Massive Stars}\\
%   Star Formation 2016, Exeter, UK, Aug, 2016 % Aug 22--26
% \item \textbf{Contributed Talk},
%   % \textit{Modeling Radiative Feedback from Massive Stars: Implementation of Adaptive Ray Tracing Method into the Athena Code}\\
%   ASTRONUM 2016 11th Annual International Conference on Numerical Modeling of Space Plasma Flows, Monterey, USA, Jun, 2016 % Jun 8
% \item Seminar Talk, \textit{Disruption of Molecular Clouds by
%     Expansion of Dusty HII Regions}\\ Star Formation/ISM Rendezvous,
%   Princeton University, USA, Oct 28, 2015
%% \item Contributed Talk, \textit{Expansion of Dusty H II Regions and
%%   Its Impact on Disruption of Molecular Clouds}, 2015 Korean
%%   Astronomical Society Spring Meeting, Seoul, Korea, Apr 17, 2015
% \item Seminar Talk, \textit{Instability of Evaporative Layers in the
%   Interstellar Medium}, Korean Numerical Astrophysics Group Meeting,
%   Daejeon, Korea, Jan 23, 2015
% \item Poster, \textit{Instability of Magnetized Ionization Fronts}\\
%   225th AAS Meeting, Seattle, USA, Jan 4--8, 2015
% \item Seminar Talk, \textit{Instability of Evaporative Layers in the
%     Interstellar Medium}\\ Star Formation/ISM Rendezvous, Princeton
%   University, USA, Nov 5, 2014
% \item Poster, \textit{Instability of Magnetized Ionization Fronts}\\
%   12th Asia-Pacific Regional IAU Meeting, Daejeon, Korea, Aug 18--22,
%   2014
% \item Attended, KITP program: \textit{Gravity's Loyal Opposition: The
%   Physics of Star Formation Feedback}, University of California, Santa
%   Barbara, USA, Jun 16--Jul 3, 2014
% \item Poster, \textit{Instability of Evaporation Fronts in the ISM}\\
%   Physical Processes in the ISM, MPE, Garching, Germany, Oct 21--25,
%   2013
% \item Attended, International Summer School on AstroComputing 2013:
%   \textit{Star \& Planet Formation}, University of California, Santa
%   Cruz, USA, Jul 22--Aug 9, 2013
%% \item Contributed Talk, \textit{Instability of Evaporation Fronts in
%%   the Interstellar Meidum}, 2013 Korean Astronomical Society Spring
%%   Meeting, Daecheon, Korea, Apr 11, 2013
%% \item Attended, \textit{Magnetic Fields in Astrophysics}, Asia Pacific
%%   Center for Theoretical Physics, Pohang, Korea, Nov 19--22, 2012
%% \item Seminar Talk, \textit{Gravitational Instability of Rotating,
%%   Vertically-Stratified, Polytropic Disks}, Korean Numerical
%%   Astrophysics Group Meeting, Daejeon, Korea, Nov 4, 2011
%% \item Poster, \textit{Gravitational Instability of Rotating,
%%   Vertically-Stratified, Polytropic Disks}, 2011 Korean Astronomical
%%   Society Fall Meeting, Jeju, Korea, Oct 5--7, 2011
%% \item Attended, \textit{2010 International School on Numerical
%%   Relativity and Gravitational Waves}, Asia Pacific Center for
%%   Theoretical Physics, Pohang, Korea, Jul 26--30, 2010
\end{itemize}
% \end{benumerate}

\section{Teaching Experience}
\begin{itemize}
\setlength\itemsep{0.0em}
\item {\bfseries Teaching Assistant}: \textit{Man and the Universe} (non-major course), Fall 2012 % Teaching Assistant for Prof. Jonghak Woo,  Fall 2012
%  Duties: Lectures, grading.
\item {\bfseries Teaching Assistant}: \textit{Observational Astronomy}, Spring 2012 % , Teaching Assistant for Prof. Jonghak Woo, Spring 2012
\item {\bfseries Bootcamp lecturer}: \textit{An Introduction to IDL Programming for Undergraduates}, 2014 % Feb
\end{itemize}

% \section{Competitively-Obtained Computing Time}
% \cvitemwithcomment{}{Co-I (Science PI), 1.6 M CPU-hrs on KISTI Tachyon2,\\
%   \textit{``Galactic Star Formation and Outflows Regulated by UV
%     Radiation and Supernova Feedback''}, 2017}{}
% \cvitemwithcomment{}{Co-I (Science PI), 1.2 M CPU-hrs on KISTI Tachyon2,\\
%   \textit{``Expansion of Dusty H II Regions and Its Dynamical Impact on
%     the Instestellar Medium''}, 2016}{}

% \section{Computer Skills}
% \cvitemwithcomment{Programming Language}{C/C++, MPI, Python, IDL}{}
% \cvitemwithcomment{Visualization/Software}{DDT, GDB, Git, yt, VisIt, ParaView}{}
% \cvitemwithcomment{Simulation Code}{\textit{Athena/Athena++}}{}

\section{Other Experience}
% \cvitem{K-GMT Time Allocation Committee}{2021-2022}
% \cvitem{Journal referee (ApJ and MNRAS)}{2020-}
% \cvitem{Military Service}{Weather observer in the Republic of Korea Air Force, 2005-2007}
% \textbf{Service:}
\begin{itemize}
\setlength\itemsep{0.0em}
\item {\bfseries Journal referee}: ApJ, MNRAS, A\&A, 2020--
\item {\bfseries LOC}: Bfields2024 -- Magnetic Fields from Clouds to Stars, 2024
\item {\bfseries Organizer}: NAOJ DoS-CfCA workshop, 2024
\item {\bfseries Organizer}: KASI TAG Seminar, 2022--2023
\item {\bfseries Member}: K-GMT Time Allocation Committee, 2021--2022
\item {\bfseries Organizer}: SNU Extragalactic Astronomy Journal Club, 2011--2013
% \item Part-time lecturer: % a short course on
% \textit{An Introduction to IDL Programming for Undergraduates}, 2014 % Feb
  % \href{http://astro.snu.ac.kr/~idl_lecture}{\textit{\color{dodgerblue}An Introduction
  %     to IDL Programming for Undergraduates}}, Feb 2014
\item {\bfseries Lead editor}: \textit{A quick guide to SNU astro graduate students.}, 2014
  % \href{https://github.com/astrosnu/grad_manual}{\color{dodgerblue}\textit{A quick guide
  %     to SNU astro graduate students.}}, 2014
\item {\bfseries Lecturer/Volunteer}: SNU Astronomy Open House, 2010--2014
\item {\bfseries Military Service}: Weather observer in the Republic of Korea Air Force, 2005--2007
\end{itemize}

% \begin{itemize}
% \item Military Service
% \end{itemize}
% \begin{itemize}
% \item Departmental Service:
% \cvitem{Departmental Service}{Served as a part-time lecturer in
% \textit{An Introduction to IDL Programming for Undergraduates}, Winter
% 2014}
%   \item SNU Astronomy Journal Club Coordinator, 2011--2013
%   \item Served as a part-time lecturer in \textit{An Introduction to
%       IDL Programming for Undergraduates}, Winter 2014
% \end{itemize}

\section{Academic References}

\cvlistdoubleitem{\textbf{Prof. Eve C.~Ostriker}\\
  \href{mailto:eco@astro.princeton.edu}{\color{dodgerblue}eco@astro.princeton.edu}\\
  Department of Astrophysical Sciences\\
  Princeton University\\
  +1-609-258-7240}
{\textbf{Prof. Woong-Tae Kim}\\
  \href{mailto:wkim@wkim.astro.snu.ac.kr}{\color{dodgerblue}wkim@astro.snu.ac.kr}\\
  Department of Physics and Astronomy\\
  Seoul National University\\
  +82-2-880-6769}

\rule[-0.9ex]{-2.5pt}{3ex}

\cvlistdoubleitem{\textbf{Prof. Thiem Hoang}\\
  \href{mailto:thiemhoang@kasi.re.kr}{\color{dodgerblue}thiemhoang@kasi.re.kr}\\
  Theoretical Astrophysics Group\\
  Korea Astronomy and Space Science Institute\\
  +82-42-865-3343}
{\textbf{Prof. Neal J. Evans II}\\
  \href{mailto:nje@astro.as.utexas.edu}{\color{dodgerblue}nje@astro.as.utexas.edu}\\
  Department of Astronomy\\
  The University of Texas at Austin\\
  +1-512-471-4396}

\rule[-0.9ex]{-2.5pt}{3ex}

\cvlistdoubleitem{\textbf{Prof. Takashi Hosokawa}\\
  \href{mailto:hosokawa@tap.scphys.kyoto-u.ac.jp}{\color{dodgerblue}hosokawa@tap.scphys.kyoto-u.ac.jp}\\
  Department of Physics\\
  Kyoto University\\
  +81-75-753-3840}{}

% \section{Experience}
% \subsection{Vocational}
% \cventry{year--year}{Job title}{Employer}{City}{}{General description no longer than 1--2 lines.\newline{}%
% Detailed achievements:%
% \begin{itemize}%
% \item Achievement 1;
% \item Achievement 2, with sub-achievements:
%   \begin{itemize}%
%   \item Sub-achievement (a);
%   \item Sub-achievement (b), with sub-sub-achievements (don't do this!);
%     \begin{itemize}
%     \item Sub-sub-achievement i;
%     \item Sub-sub-achievement ii;
%     \item Sub-sub-achievement iii;
%     \end{itemize}
%   \item Sub-achievement (c);
%   \end{itemize}
% \item Achievement 3.
% \end{itemize}}
% \cventry{year--year}{Job title}{Employer}{City}{}{Description line 1\newline{}Description line 2}
% \subsection{Miscellaneous}
% \cventry{year--year}{Job title}{Employer}{City}{}{Description}

% \section{Interests}
% \cvitem{hobby 1}{Description}
% \cvitem{hobby 2}{Description}
% \cvitem{hobby 3}{Description}

% \section{Extra 1}
% \cvlistitem{Item 1}
% \cvlistitem{Item 2}
% \cvlistitem{Item 3. This item is particularly long and therefore normally spans over several lines. Did you notice the indentation when the line wraps?}

% \section{References}
% \begin{cvcolumns}
%   % \cvcolumn{Category 1}{\begin{itemize}\item Person 1\item Person 2\item Person 3\end{itemize}}
%   % \cvcolumn{Category 2}{Amongst others:\begin{itemize}\item Person 1, and\item Person 2\end{itemize}(more upon request)}
%   % \cvcolumn[0.5]{All the rest \& some more}{\textit{That} person, and \textbf{those} also (all available upon request).}
% \end{cvcolumns}


% Publications from a BibTeX file using the multibib package
%\section{Publications}
%\nocitebook{book1,book2}
%\bibliographystylebook{plain}
%\bibliographybook{publications}                   % 'publications' is the name of a BibTeX file
%\nocitemisc{misc1,misc2,misc3}
%\bibliographystylemisc{plain}
%\bibliographymisc{publications}                   % 'publications' is the name of a BibTeX file

% \clearpage
% %-----       letter       ---------------------------------------------------------
% % recipient data
% \recipient{Company Recruitment team}{Company, Inc.\\123 somestreet\\some city}
% \date{January 01, 1984}
% \opening{Dear Sir or Madam,}
% \closing{Yours faithfully,}
% \enclosure[Attached]{curriculum vit\ae{}}          % use an optional argument to use a string other than "Enclosure", or redefine \enclname
% \makelettertitle

% Lorem ipsum dolor sit amet, consectetur adipiscing elit. Duis ullamcorper neque sit amet lectus facilisis sed luctus nisl iaculis. Vivamus at neque arcu, sed tempor quam. Curabitur pharetra tincidunt tincidunt. Morbi volutpat feugiat mauris, quis tempor neque vehicula volutpat. Duis tristique justo vel massa fermentum accumsan. Mauris ante elit, feugiat vestibulum tempor eget, eleifend ac ipsum. Donec scelerisque lobortis ipsum eu vestibulum. Pellentesque vel massa at felis accumsan rhoncus.

% Suspendisse commodo, massa eu congue tincidunt, elit mauris pellentesque orci, cursus tempor odio nisl euismod augue. Aliquam adipiscing nibh ut odio sodales et pulvinar tortor laoreet. Mauris a accumsan ligula. Class aptent taciti sociosqu ad litora torquent per conubia nostra, per inceptos himenaeos. Suspendisse vulputate sem vehicula ipsum varius nec tempus dui dapibus. Phasellus et est urna, ut auctor erat. Sed tincidunt odio id odio aliquam mattis. Donec sapien nulla, feugiat eget adipiscing sit amet, lacinia ut dolor. Phasellus tincidunt, leo a fringilla consectetur, felis diam aliquam urna, vitae aliquet lectus orci nec velit. Vivamus dapibus varius blandit.

% Duis sit amet magna ante, at sodales diam. Aenean consectetur porta risus et sagittis. Ut interdum, enim varius pellentesque tincidunt, magna libero sodales tortor, ut fermentum nunc metus a ante. Vivamus odio leo, tincidunt eu luctus ut, sollicitudin sit amet metus. Nunc sed orci lectus. Ut sodales magna sed velit volutpat sit amet pulvinar diam venenatis.

% Albert Einstein discovered that $e=mc^2$ in 1905.

% \[ e=\lim_{n \to \infty} \left(1+\frac{1}{n}\right)^n \]

% \makeletterclosing

% %\clearpage\end{CJK*}                              % if you are typesetting your resume in Chinese using CJK; the \clearpage is required for fancyhdr to work correctly with CJK, though it kills the page numbering by making \lastpage undefined
\end{document}


%% end of file `template.tex'.

%%% Local Variables:
%%% mode: latex
%%% TeX-master: t
%%% End:
